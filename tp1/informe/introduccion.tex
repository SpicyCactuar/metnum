"¿Qu\'e equipo/qui\'en crees que gana hoy?", una simple pregunta que es dif\'icil de responder con certeza dada la cantidad de aspectos que ofrecen los deportes en general. Antes de analizar una manera de responder la misma con datos y fundamentos te\'oricos, abocamos en algunos casos en los cuáles esto ser\'ia \'util:

\begin{itemize}
\item \textbf{Inversiones:} Convencer de que fortalecer financieramente a una entidad/un jugador es seguro.
\item \textbf{Puntuaci\'on:} En determinados deportes o contextos, vencer a distintos oponentes no siempre genera la misma cantidad de puntos. Esto ayuda a determinar una valuaci\'on de los mismos.
\item \textbf{Medici\'on:} Podr\'ia aportar m\'etricas internas para determinar c\'omo se encuentra la entidad/el jugador comparativamente con los dem\'as.
\end{itemize}

Proveer metodolog\'ias que incorporen aspectos varios y relevantes de los encuentros es esencial para que incremente su precisión. Dicho esto, el m\'etodo central que utilizaremos es el \textbf{\textit{Colley Matrix Method}}[1].

\subsection{Colley Matrix Method}

Para explicar el m\'etodo, haremos uso de la notaci\'on introducida en el paper con su debida aclaraci\'on cuando creamos que sea necesaria.

\subsubsection{Eliminaci\'on Gaussiana}

\textcolor{red}{\textbf{TODO: Escribir, no olvidar de mencionar por qu\'e y para que se utiliza.}}

\subsubsection{Factorizacio\'on de Cholesky}

\textcolor{red}{\textbf{TODO: Escribir, no olvidar de mencionar por qu\'e y para que se utiliza.}}