"¿Qu\'e equipo/qui\'en crees que gana hoy?", una simple pregunta que es dif\'icil de responder con certeza dada la cantidad de aspectos que ofrecen los deportes en general. Antes de analizar una manera de responder la misma con datos y fundamentos te\'oricos, abocamos en algunos casos en los cuáles esto ser\'ia \'util:

\begin{itemize}
\item \textbf{Inversiones:} Convencer de que fortalecer financieramente a una entidad/un jugador es seguro.
\item \textbf{Puntuaci\'on:} En determinados deportes o contextos, vencer a distintos oponentes no siempre genera la misma cantidad de puntos. Esto ayuda a determinar una valuaci\'on de los mismos.
\item \textbf{Medici\'on:} Podr\'ia aportar m\'etricas internas para determinar c\'omo se encuentra la entidad/el jugador comparativamente con los dem\'as.
\end{itemize}

Proveer metodolog\'ias que incorporen aspectos varios y relevantes de los encuentros es esencial para que incremente su precisión. Dicho esto, el m\'etodo que utilizaremos es el \textbf{\textit{Colley Matrix Method}} \textbf{[1]}.

\subsection{Colley Matrix Method}

Este m\'etodo busca obtener las probabilidades de que cada equipo de una liga gane su pr\'oximo encuentro, teniendo en consideraci\'on el schedule que atraves\'o cada uno de ellos (jugar contra los mejores equipos al inicio no indica que vaya a perder contra los peores en los subsiguietntes partidos), sin importar la diferencia de la cantidad de partidos jugados por cada equipo y solo considerando si el equipo gan\'o o perdi\'o (no la diferencia en puntajes obtenidos en los mismos). Es necesario notar que s\'olo aplica a modelos de competencias que no admiten empate como un resultado posible, como los que analizaremos en este contexto.

Definamos primero el modelo utilizado para el problema. Sea $\Gamma = \{1,2,...,T\}$ el conjunto de participantes de la competencia. Luego, para cada equipo \bm{$i \in \Gamma$} denominamos \bm{$n{_i}$} all n\'umero total de partidos jugados por el equipo $i$, \bm{$w{_i}$} al n\'umero de partidos ganados por el equipo $i$ y, an\'alogamente, \bm{$l{_i}$} a la cantidad de encuentros por el equipo $i$. Por \'ultimo, dados $i, j \in \Gamma$, $i \neq j$, \bm{$n{_i}{_j}$} al n\'umero de enfrentamientos entre $i$ y $j$. Una vez definido esto, a través de una serie de postulados y argumentos matemáticos, el paper \textbf{[1]} plantea que las probabilidades se obtienen como resultado de un sistema de ecuaciones lineales de la forma \bm{$Cr = b$}. \\

Donde:

\begin{itemize}
\item 
$C \in R^{T \times T}, C{_i}{_j} =
\left\{
	\begin{array}{lcc}
		-n{_i}{_j} & si & i \neq j \\
		\\ 2 + n{_i} & si & i = j \\
	\end{array}
\right.$
\item $r \in R^{T}$, donde $r{_i} = $ probabilidad de que el equipo i gane su siguiente partido
\item $b \in R^{T}$, donde $b{_i} = 1 + (w{_i} - l{_i}) / 2$
\end{itemize}

Por lo tanto, lo que se busca despejar son los elementos del vector $r$.

$C$ se denomina la \textbf{matriz de Colley} que particularmente, por lo demostrado en \textbf{[1]}, es \textbf{sim\'etrica} ($A = A{^t}$) y \textbf{definida positiva} (\textcolor{red}{DEFINITION HERE}).

Para resolver este sistema, usaremos dos algoritmos distintos para obtener sistemas de f\'acil resoluci\'on por \textit{back-substitution} y \textit{forward-substitution}. \textit{Eliminaci\'on Gaussiana} y \textit{Factorizaci\'on de Cholesky}.

\subsubsection{Eliminaci\'on Gaussiana}

La eliminaci\'on gaussiana es un algoritmo que transforma un sistema de ecuaciones en un sistema equivalente, con la caracter\'istica de que este nuevo sistema es triangular superior.

Esto se logra a través de operaciones que no alteran el conjunto solución de un sistema:

\begin{itemize}
\item Multiplicar una ecuación por un escalar
\item Intercambiar ecuaciones
\item Sumar a una ecuación con un múltiplo de otra
\end{itemize} 

Luego se resuelve por back-substitution y obtenemos el resultado deseado. Sea $A \in R^{nxn}, n \in N$. El sistema $Ax = b$ se transforma en uno equivalente $Ux = b'$, con $U$ una matriz triangular superior. \\

Los $x{_i}$ se obtienen de la siguiente manera: \\

$x{_i} = (b'{_i} - \sum\limits_{j = i + 1}^n u_{ij}x_{i}) / u_{ii}$ \\

Se puede observar que si $\exists \, i \in \{1, ..., n\} / a_{ii} = 0$ entonces no se puede realizar este procedimiento. De todas formas, en este trabajo podemos asegurar que la eliminaci\'on encuentra la $U$ y m\'as a\'un, el sistema tiene soluci\'on \'unica porque la matriz de Colley es, como se mencion\'o anteriormente, sim\'etrica y definida positiva.

Dentro del contexto de uso del \textbf{CMM}, utilizamos la Eliminaci\'on Gaussiana para obtener los $r_i$.

\subsubsection{Factorizacio\'on de Cholesky}

La factorizaci\'on de Cholesky es un caso particular de una factorizaci\'on LU, con L matriz triangular inferior y U matriz triangular superior. Bajo la hip\'otesis de este trabajo sobre las caracter\'isticas de la matriz de Colley, podemos afirmar que existe una factorizaci\'on de la forma LU de C, tal que $U = L{^t}$. \\

$LL{^t}{_i}{_j} =
\left\{
	\begin{array}{lcc}
		\sqrt{C{_i}{_i} - \sum\limits_{k=1}^{i-1} L{_i}{_k}^2} & si & i = j \\
		\\ \frac{1}{L{_i}{_i}}(C{_i}{_j} - \sum\limits_{k=1}^{i-1} L{_i}{_k}L{^t}{_j}{_k}) & si & i \neq j \\
	\end{array}
\right.$ \\

Luego, el sistema equivalente ser\'a $LL{^t}x = b$, entonces puedo resolver $Ly = b$ por forward-substitution y luego $L{^t}x = y$ para obtener el resultado deseado por back-substitution.

\textcolor{red}{\textbf{TODO: Escribir, no olvidar de mencionar por qu\'e y para que se utiliza.}}