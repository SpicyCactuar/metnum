En el trabajo realizamos m\'ultiples experimentos que analizaron tanto las implementaciones del CMM como las aplicaciones y pudimos comparar sus caracter\'isticas con las de otros m\'etodos como el WP en situaciones de Sports Analytics.

Por un lado implementamos el CMM con dos algoritmos diferentes: la eliminaci\'on gaussiana y la factorizaci\'on de Cholesky. Pudimos ver que mientras que ambos tienen una complejidad temporal de $O(n^{3})$, en Cholesky la factorizaci\'on de la matriz no altera el t\'ermino independiente. Por lo tanto el costo de $O(n^{3})$ se realizaba una \'unica vez y luego para cada vez que se corre el algoritmo con un t\'ermino independiente diferente la complejidad era \'unicamente $O(n^{2})$. Para todo esto realizamos experimentos que reflejaron nuestras hip\'otesis, obteniendo una demostraci\'on emp\'irica.

Luego analizamos las caracter\'isticas particulares del CMM.

Por un lado hicimos comparaciones con el m\'etodo de Winning Percentage. Pudimos ver que el CMM aportaba identidad a los equipos ya que no se trata \'unicamente de ganar o perder sino que tiene en cuenta contra qui\'en se juega. Ganarle a un equipo que tiene mayor cantidad de partidos ganados tiene mayor peso que ganarle a uno que siempre perdi\'o. El WP termina generando un ranking que no representa idealmente los resultados de un torneo donde no hubo suficientes partidos.

Posteriormente indagamos si el CMM produc\'ia un ranking justo. Para esto analizamos distintas schedules con caracter\'isticas particulares. Observamos por ejemplo que no siempre los ratings generados por el algoritmo eran tan intuitivos, ya que como relaciona los ratings entre los equipos, un partido entre dos afecta el rating de muchos otros que no participaron de ese partido. Por otro lado detectamos un caso m\'as grave que se genera cuando el resultado de un encuentro entre dos equipos altera el ranking de otros dos que no participaron del mismo. De ese modo podr\'ian, por ejemplo, existir equipos que decidan perder para alterar el ranking de otros.

Por \'ultimo, creamos un algoritmo greedy que deb\'ia lograr obtener el mejor ranking posible de un equipo realizando una cierta cantidad de modificaciones de los resultados de sus partidos. El algoritmo que implementamos tiene en cuenta la caracter\'istica de CMM de que ganarle a un equipo con alto rating aporta una mayor cantidad de rating que ganarle a uno con menor. De esta forma obtenemos un resultado minimal de la cantidad de partidos que deber\'ian haber sido modificados.

De este modo creemos que la adopci\'on de un algoritmo como el CMM tiene muchas ventajas frente a otros m\'as simples como el WP. Da identidad a los equipos y genera un ranking m\'as justo (aunque con sus defectos), especialmente para casos donde hubo pocos partidos entre los equipos.