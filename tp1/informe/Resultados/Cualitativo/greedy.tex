Este apartado surge de la idea de analizar si es posible que un equipo alcance la cima del ranking, si los resultados hubieran sido distintos. Es decir, contamos con un schedule fijo, seleccionamos un equipo participante y tenemos la posibilidad de modificar los partidos perdidos del mismo a ganados. Interesa minimizar la cantidad de modificaciones de este estilo tal que el equipo seleccionado quede en primera posici\'on.

La estrategia que planteamos es la siguiente:

Sea EQUIPO* el equipo que interesa analizar

Calculamos los rankings con la entrada original.

Si ranking(EQUIPO*) es el m\'aximo retornamos la cantidad de iteraciones.

Si no, buscamos el equipo con m\'aximo ranking tal que EQUIPO* tenga alg\'un partido perdido que pueda ser modificado.

Se invierte el resultado y recalculamos los rankings con este cambio.

En caso de que sea imposible modificar un partido (ya se modificaron todos los posibles), retornamos la posicion final.

\textbf{Hip\'otesis:} Los equipos con m\'as derrotas deber\'an modificar m\'as resultados que los que tienen menos. Se espera que la cantidad de partidos ganados por el equipo modificado se aproxime a los que necesit\'o el primero para estar en esa posici\'on. 

Estos fueron los standings finales de la temporada 2014-2015 de la NBA. A la derecha figura la cantidad de partidos que cada equipo deber\'ia ganar en vez de perder para quedar en la primera posici\'on seg\'un nuestro m\'etodo. \\

\begin{tabular}{|c|c|c|c|}
\hline
Posici\'on & Equipo & Record & Ranking\\
\hline
1 & Golden State & 67-15 & 0 \\
\hline
2 & Atlanta & 60-22 & 8 \\
\hline
3 & Houston & 56-26 & 7 \\
\hline
4 & LA Clippers & 56-26 & 8 \\
\hline
5 & Memphis & 55-27 & 10 \\
\hline
6 & San Antonio & 55-27 & 11 \\
\hline
7 & Cleveland & 53-29 & 14 \\
\hline
8 & Portland & 51-31 & 13 \\
\hline
9 & Chicago & 50-32 & 18 \\
\hline
10 & Dallas & 50-32 & 13 \\
\hline
11 & Toronto & 49-33 & 18 \\
\hline
12 & Washington & 46-36 & 21 \\
\hline
13 & New Orleans & 45-37 & 19 \\
\hline
14 & Oklahoma City & 45-37 & 19 \\
\hline
15 & Milwaukee & 41-41 & 26 \\
\hline
16 & Boston & 40-42 & 27 \\
\hline
17 & Phoenix & 39-43 & 25 \\
\hline
18 & Brooklyn & 38-44 & 29 \\
\hline
19 & Indiana & 38-44 & 30 \\
\hline
20 & Utah & 38-44 & 26 \\
\hline
21 & Miami & 37-45 & 29 \\
\hline
22 & Charlotte & 33-49 & 33 \\
\hline
23 & Detroit & 32-50 & 35 \\
\hline
24 & Denver & 30-52 & 35 \\
\hline
25 & Sacramento & 29-53 & 34 \\
\hline
26 & Orlando & 25-57 & 41 \\
\hline
27 & LA Lakers & 21-61 & 44 \\
\hline
28 & Philadelphia & 18-64 & 48 \\
\hline
29 & New_York & 17-65 & 50 \\
\hline
30 & Minnesota & 16-66 & 47 \\
\hline
\end{tabular} \\

Golden State qued\'o en primera posici\'on y por lo tanto requiere 0 partidos para tener el mejor ranking. No es algo sorprendente, pues adem\'as ganaron los Play-Off. Para los dem\'as equipos, la estrategia les pide alcanzar una cantidad de partidos similar a la obtenida por Golden State.

Hay un caso interesante entre Portland, Chicago y Dallas, donde Chicago tiene el mismo record que Dallas, pero necesita 5 partidos m\'as que los otros dos equipos. Esto se debe al schedule de cada uno. Seguramente Chicago jug\'o pocos partidos frente a los equipos por encima de \'el, y probablemente haya ganado algunos. Mientras que Portland y Dallas deben haber jugado m\'as veces contras los mejor rankeados y perdido en varias oportunidades, con lo que una modifici\'on en esos partidos, les significa un salto mayor en la trepada por el ranking.

Probamos tambi\'en qu\'e sucede en el caso en que por ejemplo Golden State, ganara absolutamente todos los partidos. ¿Cu\'antos encuentros deber\'an ganar los dem\'as equipos para tener un ranking mejor?

Los resultados refuerzan, como se present\'o anteriormente, las hip\'otesis planteadas, pidi\'endole a cada equipo que ganara cerca de 80 partidos sobre 82 totales (ganar en vez de perder con Golden State para disminuir su ranking y ganarle a casi todos los dem\'as pues ellos lo hab\'ian hecho).
