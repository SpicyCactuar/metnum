Dada la participaci\'on del m\'etodo CMM en este trabajo, utilizamos los algoritmos y propiedades enunciadas en la secci\'on anterior. Tener en cuenta que los deportes y \textit{schedules} de los mismos analizados en este trabajo: no cuentan con empate ni obligan, necesariamente, a los equipos a enfrentarse una misma cantidad de veces entre s\'i. Ambos son requisitos fundamentales para este an\'alisis.

\subsection{Roles de algoritmos utilizados}

Consideremos $\Gamma = \{1,2,...,T\}$ el conjunto de equipos/jugadores junto con su \textit{schedule}, $C \in R^{T \times T}$ la matriz de Colley y $b$ que se generan en consecuencia.

Como sabemos que $C$, por lo explicado en \textbf{[1]}, es \textit{sim\'etrica definida positiva}, podemos utilizar la Eliminaci\'on Gaussiana o la factorizaci\'on de Cholesky para resolverlo sin problemas:

Ataquemos el primero. Notar que vamos a tener en cuenta la siguiente matriz $A \in R^{T \times T+1}$: \\

$A = \Big(
\begin{matrix}
C
\end{matrix}
\vert
\begin{matrix}
b
\end{matrix}
\Big)$ \\

El procedimiento consiste en iterar por cada fila de la matriz buscando dejar en 0 todas las posiciones por debajo de la diagonal.

Sea $A^{k}$ la matriz luego de sufrir $k$ pasos del procedimiento. En la primer iteraci\'on, realiza el siguiente c\'alculo: \\

$Fila_{i} = Fila_{i} - \frac{a_{i1}}{a_{11}} . Fila_{1}$ \\

De esta manera logra poner en $0$ a todas las posiciones $A_{i1}$ con $i > 1$. Luego, en la iteraci\'on k: \\

$Fila_{i}^{k} - \frac{a_{ik}^{k}}{a_{kk}^{k}} . Fila_{k}^{k}$ \\

Poniendo en $0$ las posiciones $a_{ik}$ con $i > k$. Luego, en la iteraci\'on $n-1$ se obtiene una matriz U triangular superior.

Recordemos que este algoritmo tiene un caso borde que ocurre cuando alg\'un elemento $a_{ii} = 0$, pero ese caso es salvado por el hecho de que la matriz es \textit{sim\'etrica definida positiva}. Aunque se menciono que sirve para casos \textit{sin} pivoteo, se puede extender su complemento tambi\'en: los elementos de la diagonal de $A^{1}$ son positivos, por ser \textit{definida positiva}, y por ende $a_{11}$ lo es, en consecuencia los valores que obtengo en la diagonal de $A^{2}$ son positivos puesto que es el resultado de divisi\'on de positivos. Esto se extiende a todas las iteraciones de manera inductiva.

A causa de las divisiones, sumamos pivoteo parcial. En la iteraci\'on $k$, antes de realizar las operaciones anteriores, busca la fila que tenga el valor absoluto m\'as grande $a_{ik}$, $\forall i = k, ..., T$, y la intercambia con la fila $k$. Esto se realiza con el fin de minimizar errores de redondeo que se puedan dar por trabajar con aritm\'etica finita. Al dividir por un valor más alto, generamos un número más cercano al $0$, d\'onde la densidad de valores, en punto flotante, es m\'as elevada en la representaci\'on \textit{IEEE}. \\

Vayamos ahora a la Factorizaci\'on de Cholesky. Se puede utilizar en este contexto dada la naturaleza de la matriz de Colley de ser \textit{sim\'etrica definida positiva}, lo cual asegura la existencia de una factorizaci\'on LU particular (explicado en la secci\'on \ref{intro_cholesky}).

Una \'ultima pequeña optimizaci\'on agregada fue a la hora de calcular el nuevo valor de la posici\'on $A_{ij}$, se escribe en la $A_{ji}$ para facilitar la resoluci\'on posterior del sistema equivalente, puesto que de antemano sabemos de la simetr\'ia de la matriz de Colley.

\subsection{Forward-substitution y Back-substitution}
Forward-substitution resuelve un sistema triangular inferior, mientras que Back-substitution, uno triangular superior.

Estos sistemas son f\'aciles de resolver, pues las n ecuaciones tienen de 1 a n inc\'ognitas, donde resolver la de i inc\'ognitas, permite resolver la de i+1.

\subsection{Metodolog\'ia de medici\'on}

\textcolor{red}{EXPLICAR COMO FUERON PLANTEADAS Y REALIZADAS LAS MEDICIONES EXPERIMENTALES}