Dada la participaci\'on del m\'etodo CMM en este trabajo, utilizamos los algoritmos y propiedades enunciadas en la secci\'on anterior. Tener en cuenta que los deportes y \textit{schedules} de los mismos analizados en este trabajo: no cuentan con empate ni obligan, necesariamente, a los equipos a enfrentarse una misma cantidad de veces entre s\'i. Ambos son requisitos fundamentales para este an\'alisis.

\subsection{Roles de algoritmos utilizados}

Consideremos $\Gamma = \{1,2,...,T\}$ el conjunto de equipos/jugadores junto con su \textit{schedule}, $C \in R^{T \times T}$ la matriz de Colley y $b$ que se generan en consecuencia.

Como sabemos que $C$, por lo explicado en \textbf{[1]}, es \textit{sim\'etrica definida positiva}, podemos utilizar la Eliminaci\'on Gaussiana o la factorizaci\'on de Cholesky para resolverlo sin problemas: \\

Ataquemos el primero. Notar que vamos a tener en cuenta la siguiente matriz $A \in R^{T \times T+1}$: \\

$A = \Big(
\begin{matrix}
C
\end{matrix}
\vert
b
\Big)$ \\

El algoritmo de formulado por Gauss itera por cada fila de la matriz buscando dejar en 0 todas posiciones por debajo de la diagonal.

Para esto, en la primer iteraci\'on, realiza el siguiente c\'alculo: \\

$Fila_{i} - A_{i1} . Fila_{1} / A_{11}$ \\

De esta manera logra poner en $0$ a todas las posiciones $A_{i1}$ con $i > 1$

Luego, en la iteraci\'on k: \\

$Fila_{i} - A_{ik} . Fila_{k} / A_{kk}$ \\

Poniendo en $0$ las posiciones $A_{ik}$ con $i > k$

Luego, en la iteraci\'on $n-1$ se obtiene una matriz U triangular superior.

Es necesario destacar que la matriz tiene dimensi\'on $\rm I\!R^{n \times n+1}$, porque se le incluye el vector b.

Nuestra implementaci\'on, adem\'as, realiza pivoteo parcial. En la iteraci\'on k, antes de realizar las operaciones anteriores, busca la fila que tenga el valor absoluto m\'as grande en la posici\'on k y la intercambia con la fila k. Esto se realiza con el fin de minimizar errores de redondeo que se puedan dar por la representaci\'on de los n\'umeros en la m\'aquina.

Cabe destacar que el m\'etodo no falla pues el sistema tiene soluci\'on y ning\'un paso intermedio puede dejar un 0 en la diagonal.

La Factorizaci\'on de Cholesky se puede utilizar en este contexto porque la matriz de Colley es sim\'etrica y definida positiva, lo cual asegura la existencia de una factorizaci\'on LU particular (explicado en la secci\'on \ref{intro_cholesky}).

La implementaci\'on hace las cuentas que se comentan en la misma, pero adem\'as cuando calcula el nuevo valor de la posici\'on $A_{ij}$, la escribe en la $A_{ji}$ para facilitar la resoluci\'on posterior del sistema equivalente.

\subsection{Forward-substitution y Back-substitution}
Forward-substitution resuelve un sistema triangular inferior, mientras que Back-substitution, uno triangular superior.

Estos sistemas son f\'aciles de resolver, pues las n ecuaciones tienen de 1 a n inc\'ognitas, donde resolver la de i inc\'ognitas, permite resolver la de i+1.

\textcolor{red}{EXPLICAR COMO FUERON PLANTEADAS Y REALIZADAS LAS MEDICIONES EXPERIMENTALES}