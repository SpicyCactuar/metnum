\begin{abstract}
El contexto de este trabajo es el de Sport Analytics, considerando deportes en los cu\'ales los \textit{schedules} de los mismos son complejos y probablemente asim\'etricos para los participantes. El puntapié inicial del mismo es plantear técnicas de armados de \textit{rankings} que determinen una valorizaci\'on de cada participante. Nuestro objetivo será entonces elaborar métodos para confeccionar estos rankings y análisis comparativos que permitan determinar el que mejor se asimile a la realidad.
\\
\\
\\
\indent \indent \textbf{Palabras claves} \\
\\
$\circ$ Sport Analytics\\
$\circ$ Colley Matrix Method\\
$\circ$ Eliminaci\'on Gaussiana\\
$\circ$ Factorizaci\'on de Cholesky\\
\end{abstract}
