\begin{abstract}
El contexto de este trabajo es el de Sport Analytics, considerando deportes en los cu\'ales los \textit{schedules} de los mismos son complejos y probablemente asim\'etricos para los participantes. El puntapié inicial del mismo es plantear técnicas de armados de rankings basados en la probabilidad de ganar el próximo encuentro de cada equipo. Nuestro objetivo será entonces elaborar métodos para confeccionar estos rankings y realizar análisis comparativos que permitan determinar el que mejor se asimile a la realidad.\\
\textbf{\textcolor{red}{FALTA AGREGAR CONCLUSIONES}}
\\
\\
\\
\indent \indent \textbf{Palabras claves} \\
\\
$\circ$ Sport Analytics\\
$\circ$ Colley Matrix Method\\
$\circ$ Eliminaci\'on Gaussiana\\
$\circ$ Factorizaci\'on de Cholesky\\
\end{abstract}
