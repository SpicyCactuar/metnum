Aqu\'i pondremos algunos resultados/planteos interesantes.

\subsection{Veredicto de \textit{Kaggle}}

El primero fue que corrimos nuestros algoritmos para los valores ``\'optimos'' que obtuvimos de nuestro an\'alisis, utilizando la base de train completa para entrenar, y luego etiquetamos las im\'agenes provistas en la competencia de Kaggle como test.

\begin{table}[h!]
\begin{center}
\begin{tabular}{|c|c|c|c|}
	\hline
	M\'etodo & \#Autovalores & \#Vecinos & Hit-Rate \\
	\hline
	PCA & 50 & 4 & 0.97214 \\
	\hline
	PLS-DA & 16 & 10 & 0.95971 \\
	\hline
	PCA & 75 & 4 & 0.97300 \\
	\hline
	PLS-DA & 18 & 10 & 0.96186 \\
	\hline
	PCA & 85 & 4 & 0.97243 \\
	\hline
	PLS-DA & 50 & 10 & 0.97143 \\
	\hline
\end{tabular}
\end{center}
\caption{Corridas en Kaggle}
\end{table}

Hab\'iamos obtenido como \'optimos en relaci\'on $Tiempo-Hit Rate$ los primeros dos valores de la tabla. Pero dado que la competencia la gana el mayor Hit Rate, decidimos realizar una corrida con los par\'ametros de las siguientes entradas de la tabla anterior.

El mayor valor obtenido fue con el m\'etodo PCA con alpha = 75 y k = 4.

PLS-DA corrido con gammas m\'as grandes, aumentan el Hit Rate, pero el tiempo de c\'omputo es abismalmente mayor.

\subsection{Alternativa de clasificaci\'on: \textit{Promedio} de im\'agenes}

Una idea que surgi\'o en pleno trabajo fue: ``¿Habr\'a alguna manera que sea simult\'aneamente simple y efectiva de clasificar?'' Con esta pregunta en mente, calculamos el promedio de cada d\'igito de la base de \textit{train} completa y observamos los resultados: \\

\newpage
\begin{table}[h!]
\begin{tabular}{|c|c|c|c|}
	\hline
	\includegraphics[scale=4.0]{exp6/digit0} &
	\includegraphics[scale=4.0]{exp6/digit1} &
	\includegraphics[scale=4.0]{exp6/digit2} &
	\includegraphics[scale=4.0]{exp6/digit3} \\
	\hline
\end{tabular}
\begin{tabular}{|c|c|c|c|}
	\hline
	\includegraphics[scale=4.0]{exp6/digit4} &
	\includegraphics[scale=4.0]{exp6/digit5} &
	\includegraphics[scale=4.0]{exp6/digit6} &
	\includegraphics[scale=4.0]{exp6/digit7} \\
	\hline
\end{tabular}
\begin{tabular}{|c|c|}
	\hline
	\includegraphics[scale=4.0]{exp6/digit8} &
	\includegraphics[scale=4.0]{exp6/digit9} \\
	\hline
\end{tabular}
\end{table}

Esto nos dice algunas cosas: \\

Primero que todo, que la base de entrenamiento sobre la cu\'al trabajamos es abarcativa. El hecho de, a partir de haber calculado los promedios sobre la misma, hayamos obtenido im\'agenes que tienen gran similitud con potenciales n\'umeros manuscritos lo respalda.

Otra observaci\'on/conjetura que nos surgi\'o es que, probablemente, este conjunto de im\'agenes podr\'ia ser utilizado para realizar un algoritmo clasificador m\'as simple que conste de calcular la \textit{distancia} una im\'agen a clasificar contra este conjunto particular, y \textit{taggear} en base a la que se encuentre mas cercana. Este procedimiento, en comparaci\'on a los m\'etodos postulados en el trabajo, es de una simpleza notablemente mayor. De ser efectivo este algoritmo, su simpleza podr\'ia ser \'util en contextos d\'onde no se tenga f\'acil acceso a algoritmos como \textit{PCA} o \textit{PLS-DA}, como competencias, lenguages de programaci\'on poco maduros o con poca orientaci\'on cient\'fica, etc.

De todas formas, la pregunta planteada no fue del todo respondida puesto que no tenemos datos que la fundamenten. Queda pendiente obtener m\'etricas de calidad en base a este planteo y llegar a una conclusi\'on en base a los resultados.

De ser efectivo, esto podr\'ia disparar otras potenciales preguntas como: ``¿Lo ser\'a tambi\'en en otros problemas de clasificaci\'on con datos \textit{menos} acotados como clasificaci\'on de caras?'', ``¿Se podr\'a \textit{simular} la caligraf\'ia de una persona a partir de los \textit{promedios} de sus s\'imbolos lexicogr\'aficos manuscritos?'', ``¿De qu\'e depende su efectividad?''. Queda para el interesado experimentar con estos disparadores.

