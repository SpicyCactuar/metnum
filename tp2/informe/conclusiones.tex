\textbf{La conclusi\'on mas notable es la \textit{gran} reducci\'on de tiempos de ejecuci\'on, con alta efectividad, que proveen los algoritmos \textit{PCA} y \textit{PLS-DA}}. Las gr\'aficas mostradas son contundentes en este sentido. Esto es de vital importancia, sobre todo por el hecho de que se podr\'ia aplicar, casi sin esfuerzo, a d\'igitos manuscritos de mayor resoluci\'on y calidad (lo que genera un aumento en la cantidad de p\'ixeles) y solventar el incremento fuerte que sufrir\'ia el costo temporal de aplicar \textit{kNN} plano.

A su vez, \textbf{pudimos comprobar que, por lo menos para el problema planteado, existen ``representantes'' de la informaci\'on que distinguen los elementos entre s\'i}. Si bien estos representantes se obten\'ian en \textit{un nuevo espacio de los datos}, los mismos estaban intr\'insecamente relacionados con la entrada original. Esto se puede ver, por ejemplo, en los \textit{autod\'igitos} obtenidos (\ref{resultados_new-space}). Inclusive hubo casos (\ref{resultados_knn}) d\'onde introducir una mayor cantidad de variables puede disminuir (aunque sea levemente) la precisi\'on de la clasificaci\'on. Esto se debe a que a mayor informaci\'on a ser tenida en cuenta, mayores son las chances de que esa \textit{data} sea pose\'ida por 2 o m\'as elementos y por ende generar m\'as similitud entre los elementos transformados, que se traduce en confusiones a la hora de clasificar.

\textbf{Es claro que un elemento fundamental que habilita la buena aplicaci\'on del algoritmo es la base de datos sobre la cu\'al se trabaja}. Por ende, la importancia a la hora de plantear esta metodolog\'ia de clasificaci\'on yace en 2 frentes: la correcta y eficiente algoritmia, y un conjunto de \textit{entrenamiento} abarcativo y representativo. No puede flaquear ninguno de los 2 elementos a la hora de construir una resoluci\'on de este estilo.

De todas formas, \textbf{tal vez lo m\'as interesante sea} el concepto general detr\'as de los m\'etodos: \textbf{poder generar un nuevo universo de informaci\'on, relacionada con la original, en el cu\'al se pueda facilitar la ponderaci\'on de la informaci\'on y reducir la dimensi\'on de la entrada para agilizar su clasificaci\'on}. Esto, probablemenete, trascienda el contexto de este trabajo y sea aplicable a problemas de caracter\'isticas similares. Encontramos este \textit{approach} ciertamente contra-intuitivo lo que magnif\'ica, por lo menos a nuestro entender, su efectividad.