El problema de reconocer e identificar im\'agenes es uno que, a simple vista, uno creer\'ia que depende necesariamente del ojo humano. El simple hecho de plantearse c\'omo una computadora pudiese decidir algo del estilo ``esta im\'agen es de un perro'' parece de por s\'i incre\'ible e, inclusive, extremadamente \textit{dif\'icil}.

Lo interesante de las problem\'aticas es que uno puede \textit{encararlas} de una u otra manera que puede abrir el camino para resolverlo. En el caso de asociaci\'on de im\'agenes, uno puede pensarlo como \textit{reconocer} o, en contraposici\'on, \textit{diferenciar} una ilustraci\'on. En este trabajo vamos a atacar el problema (no este sino uno m\'as \textit{acotado}), por ambos caminos para poder arribar a una conclusi\'on.

Dicho esto, poder decidir para \textit{cualquier} tipo de im\'agen, sin ning\'un contexto, a qu\'e objeto o concepto est\'a asociado sigue siendo de dificultad elevada. El an\'alisis del contexto es esencial para cualquier tipo de problema y este no es la excepci\'on.

La situaci\'on concreta a ser abarcada en este trabajo es la de reconocer d\'igitos en im\'agenes con n\'umeros manuscritos con un formato particular:

\begin{itemize}
\item El n\'umero escrito debe ser del 0 al 9
\item El tamaño es de 28 $\times$ 28 p\'ixeles
\item Est\'a en escala de grises de valores entre 0 (\textit{negro}) y 255 (\textit{blanco})
\end{itemize}

Uno podr\'ia argumentar que es un formato acotado, pero si uno digitaliza un texto con n\'umeros manuscritos, en general puede partirlo en im\'agenes con las condiciones pedidas. Hay escenarios en los cu\'ales este problema tiene relevancia, por nombrar algunos:

\begin{itemize}
\item Reconocimiento de textos antiguos
\item Validaci\'on digital de textos
\item Reconocimiento de caligraf\'ia
\end{itemize}

Vamos a considerar las im\'agenes, siendo $n \in \mathbb{N}$ la cantidad, como $x_{i} \in \mathbb{R}^{m}$ con $m = 28 \times 28 = 784$ y $i \in \{1, ..., n\}$. Como est\'an en escala de grises, adem\'as se cumple que $x_{i}^{(j)} \in \{0, ..., 255\}$, d\'onde $x_{i}^{(j)}$ es el $j$-esimo elemento de $x_{i}$, $\forall j \in \{1, ..., 784\}$. En t\'erminos coloquiales, una \textit{tira} de 784 valores que est\'an en el rango de 0 a 255.

Ahora ahondaremos en los fundamentos te\'oricos de los \textit{m\'etodos num\'ericos} que servir\'an de base para el desarrollo de este trabajo.

\subsection{k Vecinos M\'as Cercanos \textit{(kNN)}}

Un primer \textit{approach} planteado es el de \textit{\textbf{k-Nearest-Neighbours}}. Como modelos tenemos: $C$ un conjunto de clases, $Y = \{y_{1}, ..., y_{n}\}$ y $Z = \{z_{1}, ..., z_{n}\}$, $y_{i}$, $z_{i} \in \mathbb{R}^m$, dos conjuntos de elementos que pueden ser clasificados o \textit{taggeados} como elementos de $C$ con la particularidad de que conocemos la clasificaci\'on de elementos del conjunto $Z$. Por lo tanto, el problema va a ser \textit{taggear} el conjunto $Y$ en base a los de $Z$. El algoritmo es simple:

Consideramos $y \in Y$, y una funci\'on $d : Y \times Z \mapsto \mathbb{R}_{> 0}$ que denominaremos \textbf{funci\'on de distancia}. Esta funci\'on toma un elemento de $Y$ y otro de $Z$, y devuelve un valor que \textit{resume} las diferencias entre un par\'ametro y el restante a un valor. Utilizaremos la misma para calcular las ``distancias'' de $y$ contra todos los elementos $z \in Z$, quedandonos s\'olo con los $k$ de menor valor. Estos ser\'an los que m\'as se \textit{parezcan} a $y$, llamamos a este conjunto $K$.

Una vez realizado esto, sea $f : C \times Z' \subset Z \mapsto \mathbb{N}$ una funci\'on que nos devuelve la cantidad de veces que se encuentra un \textit{tag} dentro de un sub-conjunto de $Z$, tomamos $c = argmax_{c \in C} f(c, K)$. De esta forma, ``asignamos'' la clase $c$ a $y$.

Este algoritmo efectivamente realiza una \textbf{clasificaci\'on supervisada}, aprendiendo de una base de ``entrenamiento'' para categorizar los elementos restantes.

De todas formas, lo simple del algoritmo tiene su contraparte en la variable tiempo de ejecuci\'on. Uno debe aplicar $d$ sobre todos los elementos de $Z$, con lo cu\'al si el conjunto es grande, dependemos de la complejidad temporal la funci\'on de distancia. La clave de este procedimiento es elegir una funci\'on que resuma las diferencias entre elementos m\'as precisamente, minimizando el computo, lo cu\'al no es una tarea f\'acil ya que, en general, funciones de distancia livianas representan menor calidad de resultados.

En respuesta a esa problem\'atica, planteamos 2 m\'etodos denominados \textbf{An\'alisis por Componentes Principales} (\textbf{PCA}, por sus siglas en ingl\'es) y \textbf{An\'alisis Discriminante por Cuadrados M\'inimos Parciales} (\textbf{PLS-DA}). \textcolor{red}{DESDE ACA NO ENTENDI NADA JAJAJA} Conceptualmente, consisten en ponderar la informaci\'on para decidir qu\'e \textit{variables} almacenan m\'as informaci\'on y reducir el an\'alisis a comparar esas variables entre los distintos elementos. La diferenciaci\'on entre las mismas va a perimitir categorizar nuestros elementos de entrada con mayor eficiencia.

\subsection{An\'alisis por Componentes Principales (\textit{PCA})}

Sean $x_{i} \in \mathbb{R}^{m}$ con $i = 1, ..., n$, muestras de $m$ variables aleatorias, con $\mu = (x_{1} + ... + x_{n}) / n \in \mathbb{R}^{m}$ la media entre las mismas. Notar que $\mu_{i} = (x_{1}^{(i)} + ... + x_{n}^{(i)}) / n$ es el promedio de la variable aleatoria $i$.

En la b\'usqueda de las variables que m\'as informaci\'on almacenen, buscamos simult\'aneamente las que posean la mayor varianza y covarianza nula. Para ello, comenzamos calculando la matriz de covarianzas $M_{X}$. Una manera de hacerlo es matricialmente, generando una matriz $X \in \mathbb{R}^{n \times m}$ con $n$ filas consistiendo de $(x_{i} - \mu)^t / \sqrt{n - 1}$ y consiguiendo el resultado deseado de la forma $M_{X} = X^{t}X$. Verificar que $M_{X}$ es la matriz de covarianzas es ver que en la posici\'on $(j, k)$ obtengo $\sigma_{jk}$, la covarianza entre $x_{j}$ y $x_{k}$: \\

\textbf{\textcolor{red}{// TODO: Chequear esta cuenta}}

$M_{X}^{(j,k)} = (x_{j} - \mu_{j})^t(x_{k} - \mu_{k}) / \sqrt{n - 1}\sqrt{n - 1} = \sum\limits_{i = 1}^{n}(x_{j}^{(i)} - \mu_{j})(x_{k}^{(i)} - \mu_{k}) / (n - 1) = \sigma_{jk}$.

Observamos adem\'as que $\sigma_{jj} = \sum\limits_{i = 1}^{n}(x_{j}^{(i)} - \mu_{j})^{2} / (n - 1) = \sigma_{j}^{2}$. Dejamos en manos del lector verificar que $M_{X}^{(j,k)} = M_{X}^{(k,j)}$. Con lo cu\'al obtuvimos una matriz que contiene a las varianzas de las variables en la diagonal y las covarianzas en el resto de las posiciones, y que es sim\'etrica. Esta \'ultima observaci\'on va a ser fundamental para los pr\'oximos pasos del procedimiento.

\subsection{An\'alisis Discriminante por Cuadrados M\'inimos Parciales (\textit{PLS-DA})}

\textbf{\textcolor{red}{// TODO: Fill}}