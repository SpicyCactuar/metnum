El suelo est\'a labrado en cuestiones te\'oricas. En esta secci\'on nos avocaremos en explicar el papel que juega cada m\'etodo y, el proceso completo partiendo de las im\'agenes y concluyendo en su clasificaci\'on.

Vamos a considerar las im\'agenes, siendo $n \in \mathbb{N}$ la cantidad, como $x^{(i)} \in \mathbb{R}^{m}$ con $m = 28 \times 28 = 784$ y $i \in \{1, ..., n\}$. Al conjunto de im\'agenes lo denominamos $I$. Como est\'an en escala de grises, adem\'as se cumple que $x^{(i)}_{j} \in \{0, ..., 255\}$, d\'onde $x^{(i)}_{j}$ es el $j$-esimo elemento de $x_{i}$, $\forall j \in \{1, ..., 784\}$. En t\'erminos coloquiales, una \textit{tira} de 784 valores que est\'an en el rango de 0 a 255. Adicionalmente, el conjunto $C = \{0, ..., 9\}$ es el conjunto de clases/\textit{labels}/\textit{tags}/d\'igitos, seg\'un como los denominemos en cada ocasi\'on particular. Finalmente, por ahora, vamos a considerar una partici\'on de $I = A \cup B$, d\'onde $A$ e $B$ son dijuntos y los mismos tienen las particularidades destacadas en \ref{intro_consideraciones}.

\subsection{Metodolog\'ias de Clasificaci\'on}

\subsubsection{Clasificaci\'on \textit{naive} con kNN}

En una primera aproximaci\'on, \textbf{utilizamos el kNN como m\'etodo de clasificaci\'on a secas}, lo que es decidir a qu\'e d\'igito pertenece cada imagen del conjunto $A$. Tomamos $a \in A$ una tira que buscamos \textit{taggear}.

Recordar que, por c\'omo lo definimos en la \textit{secci\'on \ref{intro_knn}}, el algoritmo requer\'ia \textbf{una funci\'on de distancia $\mathbf{d}$}. Como modelamos con vectores, proponemos la \textbf{\textit{norma 2}}. En otras palabras, siendo $z$ y $x$ dos im\'agenes $d(z,x) = \vert\vert z - x \vert\vert_2^2 = (z - x)^{t}(z - x)$. Notar que en su forma de \textit{producto interno}, el c\'omputo no pierde precisi\'on por ser suma y multiplicaci\'on de n\'umeros entre $0$ y $255$. Adicionalmente, fue elegida por el nivel de precisi\'on que posee (en definitiva, compara una a una cada componente). 

En consecuencia de haber definido una $d$, \textbf{obtenemos el conjunto de \textit{$\mathbf{k}$ vecinos m\'as cercanos}}. Clasificar es f\'acil: se cuentan las etiquetas de cada uno de los $k$ vecinos y la que m\'as se repita, se le asigna a $a$ \footnote{En caso de empate, nos quedamos con el primero que hayamos encontrado que maximice la cantidad de apariciones}.

Desafortunadamente, \textbf{la simplicidad tiene su costo temporal en este caso}. Las im\'agenes tienen $784$ pixels, es decir, cada punto a considerar tiene $784$ componentes. Calcular la distancia de un punto de esta dimensi\'on contra \textit{$\vert B \vert$} (el tamaño de $B$) de la misma dimensi\'on suena a mucho trabajo y \textit{lo es}. Aqu\'i es d\'onde cobran valor \textbf{PCA} y \textbf{PLS-DA}.

Ambos tienen la misma idea, obtener una matriz que realice un cambio de base tal que permita quedarnos con s\'olo una \textit{porci\'on}, la de mayor contenido de informaci\'on, de la misma. Aunque por s\'i solos, no son m\'etodos de categorizaci\'on. Por ende, \textbf{estar\'an involucrados como \textit{preprocesadores} de la informaci\'on a ser servida al \textit{kNN}} (reducir las dimensiones a considerar previo a aplicar \textit{kNN}).

\subsubsection{Reducci\'on de dimensi\'on \textit{no} supervisada con \textit{PCA}}

Siguiendo la l\'inea de \textbf{PCA} (\ref{intro_PCA}), buscamos $P$ conformado por las \textit{\textbf{Componentes Principales}} que son los autovectores de $M_{X}$. Inmediatamente surge una imposici\'on en costo de c\'omputo muy elevada: Dado que $M_{X} \in \mathbb{R}^{784 \times 784}$ es sim\'etrica, posee \textit{rango completo} de autovectores. Calcular los $784$ autovectores es pesado, y ,a\'un provisto de ellos, multiplicar \textit{todas} las im\'agenes contra la matriz generada tambi\'en lo es. Esto es bloqueante, lo que busc\'abamos era \textit{reducir} el problema en dimensi\'on para alivianar el costo de c\'omputo.

Provisto de $\alpha \in \mathbb{N}$, y $n_{iter} \in \mathbb{N}$, buscamos generar la transformaci\'on $P \in R^{\alpha \times 784}$ tal que contenga $\alpha$ \textbf{\textit{Componentes Principales}} como filas.
Como dichas componentes son los autovectores, buscamos $\alpha$ autovalores y autovectores de la matriz de covarianzas $M_{X}$. Pero al tomar un n\'umero menor de componentes, se pierde informaci\'on. Por eso mismo, decidimos \textbf{buscar las que maximicen la varianza}, son las que m\'as informaci\'on poseen en el espacio de informaci\'on transformado. Recordando el fundamento del m\'etodo planteado en la secci\'on anterior, buscabamos los autovectores de $M_{X}$ dado que la diagonalizaban. Al estar diagonalizada, los autovalores son los elementos de la diagonal, que a su vez son las varianzas de las variables en \textit{nuevo espacio de datos}. Obtener los de mayor m\'odulo nos aporta la mayor cantidad de informaci\'on posible. Para esto aplica el \textbf{M\'etodo de la Potencia} (explicado en \ref{desarrollo_metodo-potencia}), iterando tantas veces como el $n_{iter}$ provisto, combinado con \textbf{Deflaci\'on} (desarrollado en \ref{desarrollo_deflacion}). Repetimos $\alpha$ veces un paso $i$, con $B^0 = M_{X}$, de: obtener $\lambda_{i}$, el $i$-\'esimo autovalor ordenas por m\'odulo, asociado a $v_{i}$, calcular $B^{i + 1} = B^{i} - \lambda_{i}v_{i}v_{i}^{t}$ y comenzar el paso $i+1$.

Con lo cu\'al nos quedamos con una matriz $P$ que posee, por filas, los $\alpha$ autovectores de $M_{X}$ que mayor informaci\'on almacenan. El siguiente paso es aplicar el \textit{cambio de base} a cada muestra $z \in Z$ y a $y$: $Py = \hat{y} \in \mathbb{R}^{\alpha}$ y $\hat{z} = Pz \in \mathbb{R}^{\alpha}$, obteniendo sus correspondientes \textit{\textbf{Transformaciones Caracter\'isticas}}. A dicha transformaci\'on la denominamos $\mathbf{tc_{PCA}}$.

\subsubsection{Reducci\'on de dimensi\'on supervisada con \textit{PLS-DA}} \label{desarrollo_PLSDA}

Presentamos un \textit{pseudo-c\'odigo} del procedimiento para luego explicar las decisiones involucradas y las condiciones correspondientes que se deben dar para su correctitud: \\

\begin{algorithm}
\begin{algorithmic}[1]
\FOR {$i \leftarrow [1..\gamma]$}
\STATE {$M_{i} \leftarrow X^{t}YY^{t}X$}
\STATE {$w_{i} \leftarrow$ autovector asociado al mayor autovalor de $M_{i}$} \COMMENT {Deber\'ia estar normalizado, si no, normalizar}
\STATE {$t_{i} \leftarrow Xw_{i}$}
\STATE {Normalizar $t_{i}$}
\STATE {$X \leftarrow X - t_{i}t_{i}^{t}X$}
\STATE {$Y \leftarrow Y - t_{i}t_{i}^{t}Y$}
\ENDFOR
\RETURN {$w_{i}$ para cada $i \leftarrow [1..\gamma]$}
\end{algorithmic}
\caption{PLS($X, Y, \gamma$)}
\end{algorithm}

El algoritmo recibe $X \in R^{n \times 784}$, la matriz de imagenes centralizadas por la media, e $Y$ un vector que \textit{mapea} cada posici\'on con la etiqueta de la im\'agen que se encuentra en la susodicha posici\'on en $X$. Dadas estas construcciones, buscamos ir obteniendo iterativamente los \textbf{autovectores dominantes} $w_{i}$ (autovectores cuyos autovalores sean dominantes en el paso $i$) sobre la matriz $M_{i}$. Notemos que $M_{i}$ es sim\'etrica en todos los pasos: siendo $X = X_{i}$, $Y = Y_{i}$ las matrices iniciales en el paso $i$, vemos que $M_{i}^{t} = (X^{t}YY^{t}X)^{t} = (Y^{t}X)^{t}(X^{t}Y)^{t} = X^{t}(Y^{t})^{t}Y^{t}(X^{t})^{t} = X^{t}YY^{t}X = M_{i}$. Como, por lo desarrollado en \ref{intro_PLSDA}, requerimos $w_{i}$ autovector dominante, utilizamos el \textbf{M\'etodo de la Potencia} (\ref{desarrollo_metodo-potencia}) para extraerlo. El vector resultado se encuentra normalizado, por ende podemos evitar el paso de normalizaci\'on siguiente a su obtenci\'on. Para finalizar la iteraci\'on, en base a $w_{i}$, $X_{i}$ e $Y_{i}$, calcular $t_{i}$ como $Xw_{i}$ para realizar el c\'omputo de $X_{i+1}$ e $Y_{i+1}$.

Las $\gamma$ repeticiones de estos calculos nos otorga $w_{1}$, ..., $w_{\gamma}$ y los utilizamos para obtener la \textbf{Transformaci\'on Caracter\'istica} de una im\'agen $x_{i}$ como $\mathbf{tc_{PLS}(x_{i})} = (w_{1}^{t}x_{i}, ..., w_{\gamma}x_{i}) \in R_{\gamma}$.

\textbf{Con este planteo, tenemos un PLS tracicional}. \textbf{La extensi\'on a PLS-DA la hacemos tomando la $\mathbf{Y}$ como una matriz que centraliza}, como a $X$, \textbf{a una matriz que tiene un $1$ en la posici\'on $(i, j)$ si la imagen $i$ tiene etiqueta $j$\footnote{Indexando desde 1} o $(-1)$ en caso contrario}.

\subsubsection{Clasificaci\'on inteligente: \textit{Transformaci\'on} + \textit{kNN}}

Tanto \textit{PCA} como \textit{PLS-DA} nos facilitan una \textit{transformaci\'on caracter\'istica} que reduce la dimensi\'on de las im\'agenes. Pero \textbf{por s\'i mismos no son \textit{clasificadores}}. Que la dimensi\'on de las im\'agenes se encuentre reducida, abre la posibilidad de utilizar el \textit{kNN} y que su ejecuci\'on se complete en tiempos razonables.

Por lo tanto, \textbf{el algoritmo completo de clasificaci\'on}, sobre una im\'agen particular $y \in Y$ a clasificar, se compone de tomar una $\mathbf{tc_{m}}$, la transformaci\'on caracter\'istica de \textit{PCA} o \textit{PLS-DA}, obtener $tc_{m}(y)$ y $tc_{m}(z_{i})$, con $z_{i} \in Z$, y finalmente utilizar el criterio de clasificaci\'on provisto por \textit{kNN}.

\subsection{Estrategias de medici\'on}

\subsubsection{Evaluaci\'on robusta con \textit{K-fold cross validation}}

El objetivo de este trabajo es decidir una clasificaci\'on para un conjunto de entrada cuya categorizaci\'on no es conocida. Dicho esto, es imposible medir la efectividad de los algoritmos planteados si no podemos determinar que la clasificaci\'on sea precisa. Por ende, decidimos partir en dos una base de entrada $I$ cuya clasificaci\'on es conocida, para generar los conjuntos $A$ y $B$ (introducidos en \ref{desarrollo_PLSDA}). Decimos que las im\'agenes de $A$ son las de \textbf{\textit{test}} y las contenidas en $B$ las de \textit{\textbf{train}}. Como ventaja,  esta decisi\'on nos permite generar una transformaci\'on en base a las im\'agenes de \textit{train}, correr los algoritmos clasificando las de \textit{test} y comparar las resultados con las etiquetas reales paraobtener m\'etricas de calidad (\ref{desarrollo_metricas}). Por otro lado, esta metodolog\'ia correlaciona fuertemente los resultados a los conjuntos de entrada. Para ello decidimos implementar una t\'ecnica llamada \textbf{\textit{$\mathbf{\mathit{K}}$-fold cross validation}}. A grandes rasgos, esta metodolog\'ia propone \textbf{\textit{generar m\'ultiples particiones}} y ejecutar las rutinas sobre ellos. De esa forma, la elecci\'on de partici\'on queda desacoplada del resultado y se puede hacer un an\'alisis m\'as objetivo de los datos obtenidos como resultado.

Para armar los conjuntos el criterio es el siguiente: seg\'un un par\'ametro $K \in \mathbb{N}_{> 0}$, que indica el n\'umero de particiones distintas a generar, se toma $\mathbf{\frac{1}{K}}$ \textbf{partes} de $I$ como \textit{test} y el resto las consideramos de \textit{train}. Por ejemplo, con $K = 10$ extraemos el $10$ \% y se lo otorgamos a $A$, y el restante $90$ \% corresponde a $B$, con $K = 5$ la misma idea pero con $20$ \% de \textit{test} y el restante como \textit{train}.

Con las particiones obtenidas, uno experimenta sobre cada una de ellas y observa la calidad de los mismos individual y/o colectivamente.

La particularidad de la t\'ecnica es que trabaja sobre una \textit{misma} base de datos. Por esa raz\'on, \textbf{hay una imposici\'on de responsabilidad fuerte para con la base ya que una poco representativa, demasiado pequeña o con baja uniformidad de los datos puede condicionar la experimentaci\'on}. Se decidi\'o tomar como $I$ \textbf{los d\'igitos manuscritos de la famosa \textit{MNIST database}}\footnote{La cu\'al se puede en \href{https://www.kaggle.com/c/digit-recognizer/data}{\textit{\textcolor{blue}{obtener gratuitamente}}}}.

\subsubsection{M\'etricas de calidad}\label{desarrollo_metricas}

Finalmente, se desarrollaron algunas m\'etricas para medir qu\'e tan buenas son las decisiones tomadas por el clasificador: \\

* \underline{\textbf{Precision:}} Es una \textbf{medida de cu\'antos aciertos relativos tiene un clasificador dentro de una clase particular}. Es decir, dada una clase $i$, la precision de dicha clase es $tp_{i} / (tp_{i} + fp_{i})$.

En la anterior f\'ormula, $tp_{i}$ son los \textit{verdaderos positivos} de la clase $i$. Es decir, muestras que realmente pertenec\'ian a la clase $i$ y fueron exitosamente identificadas como tales. En contraposici\'on, $fp_{i}$ son los \textit{falsos positivos} de la clase $i$. Son aquellas muestras que fueron identificadas como pertenecientes a la clase $i$ cuando realmente no lo eran.

Luego, la \textit{precision} en el caso de un clasificador de muchas clases, se define como \textbf{el promedio de las precision para cada una de las clases}. \\

* \underline{\textbf{Recall:}} Es una \textbf{medida de que tan bueno es un clasificador para, dada una clase particular, identificar correctamente a los pertenecientes a esa clase}. Dada una clase $i$, el recall de dicha clase es $tp_{i} / (tp_{i} + fn_{i})$.

En la anterior f\'ormula, $fn_{i}$ son los \textit{falsos negativos} de la clase $i$. Es decir, muestras que pertenec\'ian a la clase $i$ pero que fueron identificadas con otra clase.

Luego, el \textit{recall} en el caso de un clasificador de muchas clases, se define como \textbf{el promedio del recall para cada una de las clases}. \\

* \underline{\textbf{F1-Score:}} Dado que precision y recall son dos medidas importantes que no necesariamente tienen la misma calidad para un mismo clasificador, se define esta m\'etrica para \textbf{medir un compromiso entre ambas}. Se define como $2 * precision * recall / (precision + recall)$.

\subsection{Algoritmos de Utilidad}

\subsubsection{M\'etodo de la Potencia}\label{desarrollo_metodo-potencia}

Tenemos una necesidad de encontrar \textit{autovalores} con sus \textit{autovectores asociados}. Para esto utilizamos el \textbf{M\'etodo de la Potencia}. Sea $B^{n \times n}$ la matriz de entrada.

\begin{algorithm}
\begin{algorithmic}[1]
\STATE {$v \leftarrow x_{0}$}
\WHILE {No se cumpla la condici\'on de finalizaci\'on}
\STATE {$v \leftarrow Bv$}
\STATE {Normalizar $v$}
\ENDWHILE
\STATE {$\lambda \leftarrow v^{t}Bv$}
\RETURN {$\lambda, v$}
\end{algorithmic}
\caption{M\'etodo de la Potencia($B, x_{0}$, condici\'on de finalizaci\'on)}
\end{algorithm}

Este m\'etodo busca un autovector $v_{1} \in \mathbb{R}^{m}$ tal que $\vert\vert v \vert\vert_{2} = 1$ aproximando el pasado como par\'ametro iterativamente, con la particularidad de que se corresponde con el autovalor $\lambda_{1} \in \mathbb{R}$ de manera que $\lambda_{1} > \lambda_{i}$ autovalor con $i \neq 1$.
Las condiciones de convergencia son las siguientes:

\begin{itemize}
\item \textcolor{red}{// TODO: Poner condiciones}
\end{itemize}

Notar que para dimensiones grandes, la probabilidad de elegir un vector no adecuado inicial al azar es pr\'acticamente nula, de modo que el $x_{0}$ es elegido en forma aleatoria. Las matrices a las que se les aplica el m\'etodo, en este trabajo, no cumplen con \textit{todas las precondiciones}. De todas formas, en la pr\'actica, para matrices de tamaño grande y para un n\'umero elevado de extracciones, los autovalores son distintos, pero este es un argumento emp\'irico. Utilizamos este algoritmo teniendo en mente las circunstancias.

\subsubsection{Deflaci\'on}\label{desarrollo_deflacion}

Este algoritmo soslayar\'a un esquema iterativo en el cu\'al uno puede obtener autovalores y autovectores.

Sea $B \in R^{n \times n}$ una matriz que posee autovalores distintos $\lambda_{1}, ..., \lambda_{n}$ con autovectores $v_{1}, ..., v_{n}$ asociados tales que $\vert \lambda_{1} \vert > ... > \vert \lambda_{n} \vert$, en otras palabras poder ordenarlos por m\'odulo. Adem\'as se pide que los autovectores generen una base ortonormal. En la pr\'actica, no es necesario que todas las desigualdades sean estrictas. Veamos entonces que: \\

$(B - \lambda_{1}v_{1}v_{1}^{t})v_{1} = Bv_{1} - \lambda_{1}v_{1}(v_{1}^{t}v_{1}) = \lambda_{1}v_{1} - \lambda_{1}v_{1} = 0v_{1}$

$(B - \lambda_{1}v_{1}v_{1}^{t})v_{i} = Bv_{i} - \lambda_{1}v_{1}(v_{1}^{t}v_{i}) = \lambda_{i}v_{i}$ \\

Por lo tanto, la matriz $B - \lambda_{1}v_{1}v_{1}^{t}$ posee autovalores $0, \lambda_{2}, ..., \lambda_{n}$ asociados a $v_{1}, ..., v_{n}$ de tal forma que puedo repetir el proceso obteniendo $\lambda_{2}$ y $v_{2}$. Este algoritmo se acopla muy bien con el M\'etodo de la potencia puesto que el susodicho extrae el autovalor dominante y su correspondiente autovector, con lo cu\'al es perfecto para una combinaci\'on de ambos.