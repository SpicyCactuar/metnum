\section{Conclusiones}

Pudimos apreciar el poder que provee la herramienta Cuadrados M\'inimos Lineales: un artilugio matem\'atico \textit{simple} de utilizar pero \textit{no f\'acil}. \textbf{Su efectividad est\'a atada directamente a la familia de funciones} provista. Por ende, dedicar recursos a hallar una familia acorde es de vital importancia, ya que una vez hecho eso el CML otorga funci\'on con errores peque\~nos. 

En lo relevante a los ejes de estudio, fue claro que la ocurrencia de desv\'ios en viajes a\'ereos \textbf{efectivamente presenta una relaci\'on con la distancia entre origen y destino}, y temporal con \textbf{la \'epoca del a\~no}. M\'as a\'un, pudimos plantear modelos que, como poco, \textit{\textbf{muestran tendencias}} en el porcentaje de desv\'ios en funci\'on de los aspectos mencionados. Los aeropuertos, provistos de esta informaci\'on, pueden realizar un an\'alisis con mayor profundidad en los detalles t\'ecnicos, administrativos y/o relevantes a la organizaci\'on de los vuelos para poder generar cronogramas que tengan en cuenta estas tendencias para reducir su caudal.