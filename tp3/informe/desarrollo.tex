\section{Desarrollo}

El enfoque central sobre el cu\'al vamos a ahondar va a ser los desv\'ios en vuelos. En t\'erminos m\'as t\'ecnicos, utilizaremos la t\'ecnica de \textit{cuadrados m\'inimos} para hallar modelos predictivos, utilizando par\'ametros que creemos relevantes. Veremos, para cada experimento, la raz\'on que sustenta esa elecci\'on de par\'ametros y tambi\'en compararemos distintas funciones predictoras a trav\'es del \textbf{\textit{Error Cuadr\'atico Medio}}.

Particularmente, atacaremos dos ejes de estudio:

\begin{enumerate}
\item Desv\'ios de vuelos en relaci\'on a la \'epoca del a\~no.
\item Desv\'ios de vuelos en relaci\'on a la distancia entre origen y destino.
\end{enumerate}

Conocer las tendencias sobre las desviaciones podr\'ia permitir a los aeropuertos poder organizar el cronograma de vuelos con mayor eficiencia teniendo en cuenta estos resultados. M\'as a\'un, esto no s\'olo afecta la planificaci\'on propia de cada aeropuerto sino de aquellos que interact\'uen con el mismo puesto que los planes de vuelo deben ser conocidos de antemano. Saber que un vuelo tiene cierta inclinaci\'on a sufrir un desv\'io, facilita saber qu\'e nivel de alerta deben tener los aeropuertos a esta ocurrencia.

En los experimentos planteados analizaremos este fen\'omeno y su ligadura a los aspectos particulares de cada eje de estudio.